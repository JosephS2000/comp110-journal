\documentclass{article}
\usepackage[utf8]{inputenc}

\title{A Journal of Computing Machinery and Intelligence}
\author{By Joseph Spaul}

\usepackage{natbib}
\usepackage{graphicx}

\begin{document}

\maketitle

\section{Introduction}

In this journal, I will be giving the details on my research into the field of the principles of computing. 
The seminal paper I have chosen for my research is Computing Machinery and Intelligence, written by Alan Turing 
and published in 1950. I shall be using a number of additional sources for reference to help me substantiate my 
claims about the contributions, the historical context and its lasting influence on of the seminal paper I have 
chosen to research as well as its particular field of interest, as well as a wider breadth of understanding of 
the field.

\section{Key contributions to the field}

Turing opened his seminal paper with the question, “Can machines think?”, he then went on to say that this 
question is an unhelpful because the terms ‘machine’ and ‘think’ could mean different things to different 
people and the best way to find an answer to his question would be using a Gallup poll. So, in order to find 
an answer to his question as to whether machines could think, he reframed the question. 

The question turned from can machines think? to can a machine successfully imitate a human well enough so that 
someone could mistake it for be one. This method was given a name “The Imitation Game”. The game was played by 
three people and sitting in separate rooms(I shall be referring to them as A, B and The interrogator from now 
on). A would be a man and B would be a woman, the sex of the interrogator doesn’t matter. The interrogator would 
ask questions to both of the other players trying to ascertain which is which. 

The aim for A would be to make the interrogator to make the wrong choice and Bs aim would be to make sure that the  
interrogator makes the right choice, the questions and answers would be conveyed across a written medium because 
the voice of the recipient would alert the interrogator to which of the other players is which. His question 
became, “What will happen when a machine takes the part of A in this game?”. If a machine took the place of A in 
tis test, its aim would be the same, try to give answers the way a human would and try their best to make sure that 
the interrogator thinks that they are the living person and the human player is the machine. 

His thought process was, if a machine could imitate a person to a successful enough degree that it could fool 30% of 
judges, then It would have passed the test. This test is now called the Turing test and is used as one of, if not the 
benchmark for signifying artificial intelligence. In his seminal paper, Truing also discussed Digital Computers. He 
said that digital computers would consist of three parts: Store, Executive Unit and Control. He then said that the 
Digital Computer he’s described would be classified as discrete state machines, these are machines that move from one 
state to another and that only digital computers should take part in the game. He described certain digital computers as 
universal and if they were given enough storgae, given the appropriate program and sutible increased its speed of action it 
would be able to be a part of The Imitation Game. He also wrote that a machine should be able to learn like a child and be 
able to grow, in order to get better machines, this is called machine learning.

\section{Context of the paper}

The paper was written by Alan Turing an published in October of 1950. During this time there wasn’t as much of a verity of 
papers in the field as there are today. It was the dawn of a new age of computing and Turing was its herald. With this paper 
and other works of his, he helped shape the way that we view computing, machine learning and artificial intelligence, that’s 
the reason he his considered the grandfather of modern computing. 

The paper itself was met with some backlash in and outside of the computing field. For instance, John R Searle said in his 
paper “Minds , Brains and Programs” that the only machine that is capable of thought would be the human brain and that it 
would be impossible for a computer to be able to have thought because it would only be able to mimic it, not actually be able 
to have thought that it wasn’t programmed to be able to achieve by its creator.

As the paper was published in 1950, Turing was not allowed to discuss his work at Bletchley park and the Bombe machine, which 
helped crack the enigma code. As the Bombe machine was a very early form of machine learning, it’s a great pity that Turing 
wasn’t able to use it in his seminal paper as I believe it would have been a great help to show the capability of machine 
learning. 

\section{Influence of the paper}

The seminal paper Computing Machinery and Intelligence, written by Alan Turing has had a vast influence on the field of 
artificial intelligence, machine learning and computing in general. The paper has be referenced multitudes of times by many 
different papers. Some of them being “What Computers Can't Do” and “What Computers Still Can't Do”  both written by Hubert 
Dreyfus, where he stated that human intelligence was based on an unconscious process that would be very difficult to replicate 
by using the formal rules that a computer would need to to operate. 

As stated earlier, John R. Searle also didn’t agree with Truing and in his paper “Minds, Brains and Programs” countered the 
Turing test with The Chinese Room experiment. In the Chinese Room, someone with no idea how to speak Chinese would be given 
pieces of paper with Chinese written on it and the person would have to respond using a book of instructions written in 
english. Searle stated that someone could be fooled that the person in the room would be able to read and write in Chinese, 
but they would only be using the instructions given to them without any real understanding of how to speak Chinese. So, 
according to Searle, because the person in the room doesn’t really understand Chinese, a computer doesn’t really understand 
english, they’re just using a set of instructions to respond to input.

\section{Conclusion}

In conclusion, Computing Machinery and Intelligence was and still is a pioneering paper in the field of computing that had a 
profound effect on how computing is done. As well as how artificial intelligence is looked at. It is one of the first papers 
to discuss machine learning, artificial intelligence and digital computers. It introduced the Turing test that has become a 
yard stick, used to measure  how good a machine could be at playing human. It has been cited by multitudes of papers, some of 
them agreeing with him while others did not and we are still taught about Turing decades after his death. To finish off the 
journal, I will leave you with the final quote of his seminal paper “We may hope that machines will eventually compete with 
men in all purely intellectual fields. But which are the best ones to start with? Even this is a difficult decision. Many 
people think that a very abstract activity, like the playing of chess, would be best. It can also be maintained that it is 
best to provide the machine with the best sense organs that money can buy, and then teach it to understand and speak English. 
This process could follow the normal teaching of a child. Things would be pointed out and named, etc. Again I do not know what 
the right answer is, but I think both approaches should be tried.”

\bibliographystyle{plain}
\bibliography{A M. Turing Computing Machinery and Intelligence}
\bibliography{John R. Searle Minds, Brains and Programs}
\bibliography{Horn RE Parsing the Turing test}
\bibliography{Harnad S The Turing Test is not a Trick}
\bibliography{Hybert Dreyfus What Computers Can't Do}
\bibliography{Harnad, Stevan and Scherzer, Peter (2007) First, Scale Up to the Robotic Turing Test, Then Worry About Feeling. 
Proceedings of AAAI 2007 Fall Symposium on AI and Consciousness. 08 - 11 Nov 2007.}
\bibliography{Daniel Crevier AI: The Tumultuous Search for Artificial Intelligence}
\bibliography{Hybert Dreyfus What Computers Still Can't Do}
\bibliography{Saygin, A. P. (2000), Turing Test: 50 years later. Minds and Machines 10 (4): 463–518}
\end{document}
