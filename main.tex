\documentclass{article}
\usepackage[utf8]{inputenc}

\title{A Review of Computing Machinery and Intelligence}
\author{By Joseph Spaul}

\usepackage{natbib}
\usepackage{graphicx}

\begin{document}

\maketitle

\section{Introduction}

In this journal, I will be giving the details on my research into the field of the principles of computing. 
The seminal paper I have chosen for my research is Computing Machinery and Intelligence, written by Alan Turing 
and published in 1950. I shall be using a number of additional sources for reference to help me substantiate my 
claims about the contributions, the historical context and its lasting influence on of the seminal paper I have 
chosen to research as well as its particular field of interest, as well as a wider breadth of understanding of 
the field.

\section{Key contributions to the field}

Turing opened his seminal paper with the question, “Can machines think?”, he then went on to say that this 
question is an unhelpful because the terms ‘machine’ and ‘think’ could mean different things to different 
people and the best way to find an answer to his question would be using a Gallup poll. So, in order to find 
an answer to his question as to whether machines could think, he reframed the question. 

The question turned from can machines think? to can a machine successfully imitate a human well enough so that 
someone could mistake it for be one. This method was given a name “The Imitation Game”. The game was played by 
three people and sitting in separate rooms(I shall be referring to them as A, B and The interrogator from now 
on). A would be a man and B would be a woman, the sex of the interrogator doesn’t matter. The interrogator would 
ask questions to both of the other players trying to ascertain which is which. 

The aim for A would be to make the interrogator to make the wrong choice and Bs aim would be to make sure that the interrogator makes the right choice, the questions and answers would be conveyed across a written medium because 
the voice of the recipient would alert the interrogator to which of the other players is which. His question 
became, “What will happen when a machine takes the part of A in this game?”. If a machine took the place of A in 
tis test, its aim would be the same, try to give answers the way a human would and try their best to make sure that 
the interrogator thinks that they are the living person and the human player is the machine. 

His thought process was, if a machine could imitate a person to a successful enough degree that it could fool 30% of 
judges, then It would have passed the test. This test is now called the Turing test and is used as one of, if not the benchmark for signifying artificial intelligence. In his seminal paper, Truing also discussed Digital Computers. He 
said that digital computers would consist of three parts: Store, Executive Unit and Control. He then said that the 
Digital Computer he’s described would be classified as discrete state machines, these are machines that move from one 
state to another. He described certain digital computers as universal and if they were given enough storgae, given the appropriate program and sutible increased its speed of action it would be able to be a part of The Imitation Game.

\section{Context of the paper}

\section{Influence of the paper}

Use section headings to divide your report into meaningful sections.

Use BibTeX to cite your sources. Entries have already been added to \texttt{references.bib} for the papers on the reading list.
These papers cover topics such as artificial intelligence~\cite{turing1950_intelligence, knuth1975_alphabeta}, programming language design~\cite{dijkstra1968_goto}, crpytography~\cite{rivest1978_rsa}, graphics rendering~\cite{phong1975_illumination}, and collision detection~\cite{gilbert1988_gjk}.

\section{Conclusion}

See LearningSpace for the assignment brief, containing information on marking criteria and further guidance.

\bibliographystyle{plain}
\bibliography{Computing Machinery and Intelligence}
\bibliography{John R. Searle Minds, Brains and Programs}
\bibliography{Horn RE Parsing the Turing test}
\bibliography{Harnad S The Turing Test is not a Trick}
\bibliography{Hybert Dreyfus What Computers Can't Do}
\bibliography{Harnad, Stevan and Scherzer, Peter (2007) First, Scale Up to the Robotic Turing Test, Then Worry About Feeling. Proceedings of AAAI 2007 Fall Symposium on AI and Consciousness. 08 - 11 Nov 2007.}
\bibliography{Daniel Crevier AI: The Tumultuous Search for Artificial Intelligence}
\bibliography{Hybert Dreyfus What Computers Still Can't Do}
\bibliography{Saygin, A. P. (2000), Turing Test: 50 years later. Minds and Machines 10 (4): 463–518}
\end{document}
