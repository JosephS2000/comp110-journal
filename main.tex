\documentclass{article}
\usepackage[utf8]{inputenc}

\title{A Review of Computing Machinery and Intelligence}
\author{By Joseph Spaul}

\usepackage{natbib}
\usepackage{graphicx}

\begin{document}

\maketitle

\section{Introduction}

This is the template project for COMP110 Assignment 2: Research Journal.

\section{Key contributions to the field}

\section{Context of the paper}

\section{Influence of the paper}

Use section headings to divide your report into meaningful sections.

Use BibTeX to cite your sources. Entries have already been added to \texttt{references.bib} for the papers on the reading list.
These papers cover topics such as artificial intelligence~\cite{turing1950_intelligence, knuth1975_alphabeta}, programming language design~\cite{dijkstra1968_goto}, crpytography~\cite{rivest1978_rsa}, graphics rendering~\cite{phong1975_illumination}, and collision detection~\cite{gilbert1988_gjk}.

\section{Conclusion}

See LearningSpace for the assignment brief, containing information on marking criteria and further guidance.

\bibliographystyle{plain}
\bibliography{Computing Machinery and Intelligence}
\bibliography{John R. Searle Minds, Brains and Programs}
\bibliography{Horn RE Parsing the Turing test}
\bibliography{Harnad S The Turing Test is not a Trick}
\bibliography{Hybert Dreyfus What Computers Can't Do}
\bibliography{Harnad, Stevan and Scherzer, Peter (2007) First, Scale Up to the Robotic Turing Test, Then Worry About Feeling. Proceedings of AAAI 2007 Fall Symposium on AI and Consciousness. 08 - 11 Nov 2007.}
\bibliography{Daniel Crevier AI: The Tumultuous Search for Artificial Intelligence}
\bibliography{Hybert Dreyfus What Computers Still Can't Do}
\bibliography{Saygin, A. P. (2000), "Turing Test: 50 years later". Minds and Machines 10 (4): 463–518}
\end{document}
